%%%%%%%%%%%% GITHUB TEMPLATE

% Making headings start at same place at first lettter
\usepackage{titlesec}
\titlelabel{\llap{\thetitle\quad}}

%%%%% Packages that need no configuration
% Footnotes fixed to bottom
\usepackage[bottom]{footmisc}
% Allows us to use "H" positioning for floats if necessary
\usepackage{float}
% Allows us to cancel stuff
\usepackage{cancel}
\usepackage{booktabs}
\usepackage{multicol}
% TBC
\usepackage[edges]{forest}
% Allows us to have sufigures
\usepackage{subfig}
% Setting our text to not bet justified across the whole page and no hyphen
\usepackage[document]{ragged2e}
\usepackage[none]{hyphenat}
% Macros for differentials
\usepackage[thinc]{esdiff}

% Importing our tikz for making diagrams
\usetikzlibrary{trees}
\usetikzlibrary{backgrounds}

% Setting our mono code to Hasklug (must be install on system)
\usepackage{fontspec}
\setmonofont[
	Contextuals={Alternate} % Gives us ligatures
]{Hasklug Nerd Font}

% Changing chaper text
\renewcommand{\chaptername}{Chapter}

% Chaning padding between footnote line and text
\addtolength{\skip\footins}{1em}

% Decreasing space before chapter title
\titleformat{\chapter}[display]{
	\normalfont\Large\bfseries
}{\chaptertitlename\ \thechapter}{0pt}{\huge}
\titlespacing*{\chapter}{10pt}{0pt}{20pt}

% Change space between bullets
\renewcommand{\tightlist}{
	\setlength{\parsep}{1em}
	\setlength{\parskip}{0em}
	\setlength{\topsep}{0em}
	\setlength{\itemsep}{0em}
	\setlength{\partopsep}{0.8em}
}
% Chaging the second intented bulled to a circle
\renewcommand{\labelitemii}{$\circ$}

% Making the repositioning of our images more forgiving
\renewcommand{\topfraction}{.85}
\renewcommand{\bottomfraction}{.7}
\renewcommand{\textfraction}{.15}
\renewcommand{\floatpagefraction}{.66}
\setcounter{topnumber}{3}
\setcounter{bottomnumber}{3}
\setcounter{totalnumber}{4}

% fancyhdr for header and footer
\usepackage{fancyhdr}
\pagestyle{fancy}
\fancyhf{}
\fancyhead[R]{Kamal Sacranie}
\fancyhead[L]{Chapter \thechapter}
\fancyfoot[C]{\thepage}
% Changing header rule style
\renewcommand{\headrulewidth}{2pt}
\renewcommand{\footrulewidth}{0pt}


% Creating shaded indented box for quote
\usepackage{xcolor}
\usepackage[framemethod=TikZ]{mdframed}

% This is how you define a color in latex
\colorlet{quoteshadecolor}{black!5!white}
% Renewingn our quote environemnt with the new color and a line on th eleft
\renewenvironment{quote}{
	\bigskip\begin{mdframed}[
			skipabove=\topskip,
			skipbelow=\topskip,
			backgroundcolor=quoteshadecolor,
			leftmargin=0.5cm,
			rightmargin=0.5cm,
			topline=false,
			rightline=false,
			bottomline=false,
			nobreak=true,
		]\itshape%itemshape is for italics
		}{
	\end{mdframed}
}


%%%%%%%%%%%% Tcolorboxes
% Defining our shadecolor (used fo shading code blocks usually)
\definecolor{code}{RGB}{1,22,80} % This is how you define a color in latex
\colorlet{shadecolor}{code} % redefining our shadecolor to be code color

%%%%%%%% Code mdframedbox
% Using tcolorbox to make a background box
\usepackage[many, listings]{tcolorbox}
\newtcolorbox{codeboxback}{
	enhanced,
	colback=white!50!black,
	colframe=white!50!black,
	fonttitle=\bfseries,
	breakable,
	drop fuzzy midday shadow=black!30!white,
}

% This allows us to only run code if an environment exists
\ifcsmacro{Shaded}{
	% Renewing the shaded environemnt with new mdframed box
	\renewenvironment{Shaded}{
		\bigskip
		% \begin{codeboxback}%[drop fuzzy midday shadow]
		\begin{mdframed}[
				skipabove=\topskip*2,
				outerlinewidth= 0,
				linewidth=0pt,
				roundcorner= 3pt,
				backgroundcolor= shadecolor,
				outerlinecolor= shadecolor,
				innertopmargin= \topskip,
				innerbottommargin=\topskip,
				leftmargin=-0.4cm,
				rightmargin=-0.4cm
			]}{
		\end{mdframed}
		% \end{codeboxback}
		\smallskip
	}
}{}

%%%%% THE SOLUTION TO WRAPPING ENVS!!!
% Styling code output
\AddToHook{env/verbatim/before}{\begin{mdframed}[
			skipabove=\topskip*2,
			outerlinewidth= 0,
			linewidth=0pt,
			backgroundcolor= black!10!white,
			outerlinecolor= black!40!white,
			roundcorner=5pt,
			innertopmargin= \topskip,
			innerbottommargin=\topskip,
		]}
		\AddToHook{env/verbatim/after}{\end{mdframed}}

% Styling the example box
\newtcolorbox{examplebox}[2][]{
	colback=green!20!white,
	colframe=green!20!black,
	coltitle=white,
	fonttitle=\bfseries,
	colbacktitle=green!20!black,
	enhanced,
	breakable,
	grow to left by=0.5cm,
	grow to right by=0.5cm,
	left=0.7cm,
	right=0.5cm,
	attach boxed title to top center={yshift=-2mm},
	title={#2},#1
}
% Makes the examplebox handle footnotes correctly
\makesavenoteenv{examplebox}
% Enclosing our example envs in a tcolorbox
\AddToHook{env/example/before}{\begin{examplebox}{Example}}
		\AddToHook{env/example/after}{\end{examplebox}}
